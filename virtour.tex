\documentclass{sig-alternate-05-2015}

\begin{document}

% DOI
\doi{}

% ISBN
\isbn{}

%Conference
%\conferenceinfo{PLDI '13}{June 16--19, 2013, Seattle, WA, USA}

%\acmPrice{\$15.00}

%
% --- Author Metadata here ---
%\conferenceinfo{Turing Scholars Class of 2016}{Austin, Texas USA}
\conferenceinfo{College of Natural Sciences, Department of Computer Science,
Building-Wide Intelligence Lab}{Austin, Texas}
%\CopyrightYear{2007} % Allows default copyright year (20XX) to be over-ridden - IF NEED BE.
%\crdata{0-12345-67-8/90/01}  % Allows default copyright data (0-89791-88-6/97/05) to be over-ridden - IF NEED BE.
% --- End of Author Metadata ---

\title{Virtour: Telepresence system for remotely-operated building tours}

\numberofauthors{5}
\author{
% 1st. author
\alignauthor
Patricio Lankenau\\
\email{pato@cs.utexas.edu}
% 2nd. author
\alignauthor
Jivko Sinapov
\email{jsinapov@cs.utexas.edu}
% 3rd. author
\alignauthor
Matteo Leonetti
\email{m.leonetti@leeds.ac.uk}
\and 
% 4th. author
\alignauthor
Shiqui Zang
\email{szhang@cs.utexas.edu}
% 5th. author
\alignauthor
Peter Stone
\email{pstone@cs.utexas.edu}
}
\date{August 14 2016}

\maketitle
\begin{abstract}
  This is my abstract
\end{abstract}

\keywords{robots, telepresence, remote control, virtual tours}

\section{Introduction}

The University of Texas at Austin has a constant stream of visitors and tours
of the beautiful campus. Of special interest to us, are the large number of
tours given at our computer science building: the Gates Dell Complex (GDC). The
tour guests range in ages and backgrounds, and tend to be prospective students
to both undergraduate and gradute programs, or visiting faculty. Unfortunately,
there is a large population of prespective students that are unable to
physically come to our campus and are thus unable to partake in the
conventional tours.

Our lab has a group of autonomous wheeled robots which can localize, navigate,
and perform tasks without human intervention for long periods of time.
Furthermore, our lab is placed in a central part of our building and is thus a
common place for tours. As such it only made sense that we utilize the platform
we have built to try to solve the aforementioned problem.

This is why we designed Virtour. Virtour is a public facing system for
teleoperated building tours. Virtour builds on the existing Building-Wide
Intelligence autonomous robot platform and is designed to keep the robots and
any humans involved safe. Through the use of modern web and robot technologies
it allows untrained public users to remotely control our robots in what we call
a virtual tour. Our system is created to balance external control abilities
while maintaining our rigorous standard of safety and security for the robots
and people involved. As such it gives the user control of what the robot is
doing, while at the same time using existing the autonomous navigation
capabilities and obstacle avoidance.

\section{Related Work}

\section{Robot Platform}

Virtour is made to be run on the Building Wide Intelligence Project's segbot
robot platform {CITE}. These robots are designed to be fully autonomous and
cohabitate the Gates Dell Complex Computer Science building with the humans
within. The segbot robot platform has three currently operation versions. Our
last generation version 2 robots, a version 2 with an additional Kinova arm,
and our latest generation version 3 robots. Although virtour supports all three
versions, it is mostly run on the latest generation so that is what is
described.

\subsection{Hardware Platform}

The robot's base is a Segway Robotipcs Mobility Platform (RMP) {CITE}, which is
powered by a lithium-ion battery pack. The frame was designed in-house and
supports a wide array of sensors. For navigation, localization, and obstacle
avoidance, we use a Velodyne Puck lidar. Point clouds and RGB data is provided
by a Microsoft Kinect. Our latest generation robots also have a laser range
finder to compensate for the lidar's blind spots. The robot is equipped with a
custom-built computer which runs Ubuntu 14.04. The computer is powered by the
RMPs battery, thus removing the need for an external car battery (which was
present in our version 2 robots).  The battery life on a running robot is
approximately 6 hours.

\subsection{Software Stack}

Our robots are powered by the Robot Operating System (ROS) {CITE}. {GET MORE
INFO FROM SOME ONE"S PAPER}

\section{The Web Client}

Virtour consists of two platforms, the user facing client, and the server
and associated software that runs on the robots. The user client is built using
web 2.0 technologies to adhere to modern web development trends and
simulatniously support as many platforms as possible. We decided to use a
web-based client because of the increasing prominence of web browsers in
people's lives. Furthermore, a web based approach means that our end-users do
not have to install any additional software to connect with or use the robots,
thus reducing the friction for trying our service.

\subsection{Modern Approach}

The website is designed to be simple and functional while still being
aesthetically pleasing to end users. It uses a grid system, powered by
Bootstrap 2.0, to create a fully responsive web layout. This allows us to
support any web-powered platform (eg: mobile devices, tablets, and computers)
by making the website scale and re-organize based on the specifications of the
device.

When a user first visits our website, he or she is greeted by a list of our
currently active and available robots (more on server implementation later).
From here our user can select a robot to connect to (by clicking on the robot's
name and image) to initate a virtual tour session. Tour sessions can be either
led or spectated. Each tour can have at most one leader, but no limit on the
number of spectators. If the tour has no existing leader and tours 

\subsection{Leader UI}

\subsection{Guest UI}

\section{The Server}

\subsection{Tour Manager}

\subsubsection{Leader Management}

\subsubsection{Robot Control}

\subsubsection{Authentication}

\subsection{IP management}

\section{Scavenger Hunt Integration}

\section{Conclusions}

\section{Acknowledgments}

\bibliographystyle{abbrv}
\bibliography{sigproc}  % sigproc.bib is the name of the Bibliography in this case
\end{document}
