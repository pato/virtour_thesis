\documentclass{sig-alternate-05-2015}

\begin{document}

% DOI
\doi{}

% ISBN
\isbn{}

%Conference
%\conferenceinfo{PLDI '13}{June 16--19, 2013, Seattle, WA, USA}

%\acmPrice{\$15.00}

%
% --- Author Metadata here ---
%\conferenceinfo{Turing Scholars Class of 2016}{Austin, Texas USA}
\conferenceinfo{College of Natural Sciences, Department of Computer Science,
Building-Wide Intelligence Lab}{Austin, Texas}
%\CopyrightYear{2007} % Allows default copyright year (20XX) to be over-ridden - IF NEED BE.
%\crdata{0-12345-67-8/90/01}  % Allows default copyright data (0-89791-88-6/97/05) to be over-ridden - IF NEED BE.
% --- End of Author Metadata ---

\title{Virtour: Telepresence system for remotely-operated building tours}

\numberofauthors{5}
\author{
% 1st. author
\alignauthor
Patricio Lankenau\\
\email{pato@cs.utexas.edu}
% 2nd. author
\alignauthor
Jivko Sinapov
\email{jsinapov@cs.utexas.edu}
% 3rd. author
\alignauthor
Matteo Leonetti
\email{m.leonetti@leeds.ac.uk}
\and 
% 4th. author
\alignauthor
Shiqui Zang
\email{szhang@cs.utexas.edu}
% 5th. author
\alignauthor
Peter Stone
\email{pstone@cs.utexas.edu}
}
\date{August 14 2016}

\maketitle
\begin{abstract}
  This is my abstract
\end{abstract}

\keywords{robots, telepresence, remote control, virtual tours}

\section{Introduction}

\section{Related Work}

\section{Robot Platform}

Virtour is made to be run on the Building Wide Intelligence Project's segbot
robot platform {CITE}. These robots are designed to be fully autonomous and
cohabitate the Gates Dell Complex Computer Science building with the humans
within. The segbot robot platform has three currently operation versions. Our
last generation version 2 robots, a version 2 with an additional Kinova arm,
and our latest generation version 3 robots. Although virtour supports all three
versions, it is mostly run on the latest generation so that is what is
described.

\subsection{Hardware Platform}

The robot's base is a Segway Robotipcs Mobility Platform (RMP) {CITE}, which is
powered by a lithium-ion battery pack. The frame was designed in-house and
supports a wide array of sensors. For navigation, localization, and obstacle
avoidance, we use a Velodyne Puck lidar. Point clouds and RGB data is provided
by a Microsoft Kinect. Our latest generation robots also have a laser range
finder to compensate for the lidar's blind spots. The robot is equipped with a
custom-built computer which runs Ubuntu 14.04. The computer is powered by the
RMPs battery, thus removing the need for an external car battery (which was
present in our version 2 robots).  The battery life on a running robot is
approximately 6 hours.

\subsection{Software Stack}

Our robots are powered by the Robot Operating System (ROS) {CITE}. {GET MORE
INFO FROM SOME ONE"S PAPER}

\section{The Web Client}

\subsection{Modern Approach}

\subsection{Leader UI}

\subsection{Guest UI}

\section{The Server}

\subsection{Tour Manager}

\subsubsection{Leader Management}

\subsubsection{Robot Control}

\subsubsection{Authentication}

\subsection{IP management}

\section{Scavenger Hunt Integration}

\section{Conclusions}

\section{Acknowledgments}

\bibliographystyle{abbrv}
\bibliography{sigproc}  % sigproc.bib is the name of the Bibliography in this case
\end{document}
